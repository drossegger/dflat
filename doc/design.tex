The first idea was to perform algorithm selection at every \inline$clasp$ call made by \mbox{D-FLAT}. In theory this approach achieves the most performance gain since the best configuration is used for every call but since \inline$clasp$ is called mostly on small subproblems which are relatively easy to solve, the overhead generated by the algorithm selection would likely be too high such that the gains in performance cannot compensate the cost of the selection process.

Another approach is to only analyse the main problem by looking at features of the ungrounded instance and select the best configuration based on this features for all \inline$clasp$ calls. The disadvantage of this approach is that a good configuration for the problem might not be a good configuration for its subproblems but since most of the subproblems are easy to solve and big subproblems have most likely a similar structure as the problem, it might still be feasible.

The first approach needs careful implementation to work efficiently and the use of existing technologies is harder than in the second approach. Since it was also not known if algorithm selection techniques benefit \mbox{D-FLAT} at all, it was more feasible to use the second approach to investigate if the tuning of \inline$clasp$ using algorithm selection has an impact on the performance of \mbox{D-FLAT}.

